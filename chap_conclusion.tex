%------------------------------------------------------
% QUOTE
% If you really feel like adding a quote
% to your page, uncomment out the following.
%------------------------------------------------------
%\begin{savequote}[45mm]
%When theory and experiment agree, 
%that is the time to be especially suspicious. 
%\qauthor{Niels Bohr}
%\end{savequote}
%------------------------------------------------------
% QUOTE
% If you really feel like adding a quote
% to your page, uncomment out the previous.
%------------------------------------------------------


\chapter{Conclusion}

I conclude by restating the main postulate of this thesis, that ABC can offer improvements over some limiting aspect of traditional likelihood based Bayesian inference. In chapter \ref{SGE} I showed how ABC and the information available by opening the likelihood can be used to drive a more diagnostic inversion scheme. This inversion improves in comparison to traditional likelihood machinery by ensuring each model update has impact. The result is faster optimization while retaining the same formal statistical guarantees offered by probabilistic formulations of geophysical inverse problems.  Faster optimization is necessary if we wish to push the limits in the scale of geophysical problems which fundamentally require a probabilistic approach due to trade-offs between parameters, data and modelization uncertainty or non-uniqueness. This thesis is foundational in linking stochastic sampling to ideas which are generally reserved for linearized geophysical inverse problems. It is the first application of ABC in geophysics, providing an initial connection to the rapidly expanding field of likelihood-free methods. This value is highlighted by the new likelihood-free methods consistently emerging which may benefit geophysics \citep{papamakarios2016fast,song2017nice}.\par

While the result presented in this exploratory analysis is positive, a more comprehensive study is necessary. Firstly, the range of tests for ABC-tomography needs to be expanded to larger and more realistic Earth models. How robust the outcomes highlighted here are needs to be established clearly. Further exploration is also needed in how to best drive optimization given the available information, while still retaining formal statistical guarantees. The potential for comparative improvement is also not restricted to optimization. It may be worthwhile assessing the impact of simplifying assumptions about the nature of modelization and data uncertainty as well as the correlation structure present. There may also be scope for assessing the adequacy of models given the data, as opposed to relative adequacy against other models. \par
This thesis demonstrates how ABC can be used to build upon the progress made in probabilistic tomography. 
