%------------------------------------------------------
% QUOTE
% If you really feel like adding a quote
% to your page, uncomment out the following.
%------------------------------------------------------
%\begin{savequote}[45mm]
%When theory and experiment agree, 
%that is the time to be especially suspicious. 
%\qauthor{Niels Bohr}
%\end{savequote}
%------------------------------------------------------
% QUOTE
% If you really feel like adding a quote
% to your page, uncomment out the previous.
%------------------------------------------------------


\chapter{Conclusion}

I conclude by restating the main postulate of this thesis, that ABC can offer improvement over some limiting aspects of traditional likelihood based Bayesian inference. In chapter \ref{SGE} I showed how ABC can use the information available by opening the likelihood to drive a more diagnostic inversion scheme. This diagnostic inversion improves in comparison to traditional likelihood machinery by ensuring each model update has impact. The result is a more rapid convergence to low misfit models while retaining the same formal statistical guarantees offered by probabilistic formulations of geophysical inverse problems. Algorithms which rapidly find and explore low misfit models are necessary if we wish to push the limits in the resolution and scale of solvable geophysical inverse problems which fundamentally require a probabilistic approach. This thesis is foundational in linking stochastic sampling to ideas which are generally reserved for linearized geophysical inverse problems. It is the first application of ABC in geophysics, providing an initial connection to the rapidly expanding field of likelihood-free methods. This value is highlighted by the new likelihood-free methods consistently emerging which may benefit geophysics \citep{papamakarios2016fast,song2017nice}.\par

While the result presented in this exploratory analysis is positive, a more comprehensive study is necessary. Firstly, the range of tests for the diagnostic ABC inversion needs to be expanded to larger and more realistic Earth models. The robustness of the results presented here needs to be clearly established. Further exploration is also needed in how to best drive a diagnostic scheme given the available information, while still retaining formal statistical guarantees. The potential for comparative improvement is not restricted to this angle. It is worthwhile assessing the impact of simplifying assumptions about the nature of modelization and data uncertainty, compared with its true nature. There is also be scope for assessing the model adequacy given the data, as opposed to relative adequacy against other models. \par
This thesis demonstrates how ABC can be used to build upon the progress made in probabilistic tomography.
