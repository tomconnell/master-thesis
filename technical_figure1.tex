\begin{tcolorbox}[enhanced jigsaw,breakable,pad at break*=1mm,title=Technical figure 1: General likelihood derivation, title filled,fonttitle=\sffamily\bfseries,fontupper=\sffamily\scriptsize,before upper={\parindent15pt}]
%\chapter{Box 1: Likelihood derivation}
\label{tf1}

\begin{multicols}{2}
\noindent \citet[p.89-90]{gregory2005bayesian} outline constructing a likelihood, $\mathcal{L}(\bm{\theta}|\bm{y})$, from starting assumptions. Firstly, a model for the data is assumed:
\begin{equation}
y_i = z_i + e_i
\label{general_form_inference}
\end{equation}
where $e_i$ is the uncertainty in the data, and $z_i$ is the data simulation from the deterministic forward relation:
\begin{equation}
z_i = \bm{g}(\bm{\theta})
\end{equation}
Both $z_i$ and $e_i$ are represented by distributions:
\begin{equation}
p(z_i|\bm{\theta}) = f_Z(z_i)
\label{fz_dist}
\end{equation}
\begin{equation}
p(e_i|\bm{\theta}) = f_E(e_i)
\label{fe_dist}
\end{equation}
We are critically interested in arriving in an equation for $p(y_i|\bm{\theta})$, which will define $\mathcal{L}(\bm{\theta}|\bm{y})$. However, as the relationship in equation \ref{general_form_inference} specifies, $y_i$ depends on both $z_i$ and $e_i$. Hence, to arrive at $p(y_i|\bm{\theta})$ we must consider the distributions of equations \ref{fz_dist} \& \ref{fe_dist}. If we first consider the joint distribution $p(y_i,z_i,e_i|\bm{\theta})$ then our likelihood can be found by integrating out, 'marginalizing', $p(z_i|\bm{\theta})$ and $p(e_i|\bm{\theta})$ to leave a distribution for $p(y_i|\bm{\theta})$:
\begin{equation}
p(y_i|\bm{\theta}) = \int \int \text{d}z_i\ \text{d}e_i\ p(y_i,z_i,e_i|\bm{\theta})
\label{most_general_L}
\end{equation}
The definition of conditional probability allows equation \ref{most_general_L} to be rewritten as:
\begin{equation}
p(y_i|\bm{\theta}) = \int \int \text{d}z_i\ \text{d}e_i\ p(z_i,e_i|\bm{\theta})\ p(y_i|z_i,e_i,\bm{\theta})
\end{equation}
If we assume $z_i$ and $e_i$ are independent:
\begin{equation}
p(y_i|\bm{\theta}) = \int \int \text{d}z_i\ \text{d}e_i\ p(z_i|\bm{\theta})\ p(e_i|\bm{\theta})\ p(y_i|z_i,e_i,\bm{\theta})
\label{halfway_through_derivation}
\end{equation}
From the relationship in equation \ref{general_form_inference}, $y_i-z_i-e_i = 0$, $p(y_i|z_i,e_i,\bm{\theta})$ can be reasonably represented as a dirac-delta function such that:
\begin{equation}
p(y_i|z_i,e_i,\bm{\theta}) = \delta(y_i-z_i-e_i)
\label{dirac_equality}
\end{equation}
This is a natural definition, as the probability density of $y_i$ given $z_i$ and $e_i$ should be focused when the exact relation of equation \ref{general_form_inference} is met. The dirac-delta function serves this role perfectly while still exhibiting properties of a PDF, non-negative and $\int_{-\infty}^{\infty}\delta = 1$. Considering equation \ref{dirac_equality}, equation \ref{halfway_through_derivation} becomes:
\begin{equation}
p(y_i|\bm{\theta}) = \int \int \text{d}z_i\ \text{d}e_i\ p(z_i|\bm{\theta})\ p(e_i|\bm{\theta})\ \delta(y_i-z_i-e_i)
\end{equation}
Adopting the form from equations \ref{fz_dist} \& \ref{fe_dist}:
\begin{equation}
p(y_i|\bm{\theta}) = \int \text{d}z_i\ f_Z(z_i) \int \text{d}e_i\ f_E(e_i) \delta(y_i-z_i-e_i)
\label{almost_general_form}
\end{equation}
The dirac-delta function in the second integral of equation \ref{almost_general_form} serves to isolate the value/values of $f_E(e_i)$ when $e_i = y_i - z_i$. Hence, it is equivalent to consider:
\begin{equation}
\int \text{d}e_i\ f_E(e_i) \delta(y_i-z_i-e_i) = f_E(y_i - z_i)
\end{equation}
This leaves a general form for the likelihood $\pi(y_i|\bm{\theta})$ as: 
\begin{equation}
p(y_i|\bm{\theta}) = \mathcal{L}(\bm{\theta}|y_i) = \int \text{d}z_i\ f_Z(z_i)\ f_E(y_i - z_i)
\label{general_likelihood}
\end{equation}
However this general equation is not in a form which can be practically applied to inference problems. To achieve a practical form further assumptions must be made about the distributions $f_Z(z_i)$ and $f_E(e)$. This is done by assigning parametric distributions. As we will see in the next section, assigning Gaussian distributions will allow \ref{general_likelihood} to form a usable expression which will be our applied-form likelihood.
\end{multicols}
\end{tcolorbox}