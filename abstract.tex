\chapter{Abstract}

%Geophysical inverse problems open up the interior of the Earth. Their solution describes the inner working of our planet. The methods of inverse problems are a continually evolving field, fed by breakthroughs in mathematics. Recently, probabilistic solutions have been advocated for in geophysics. Probabilistic tomography can tackle the inherit non-uniqueness of the model space and the uncertainty in both data and modelization. This formulation depends on traditional likelihood machinary. By extension, it also suffers from the limiting aspects and assumptions which are implicit in the construction of a likelihood function. Here I explore the geophysical applicability of likelihood-free probabilistic methods, coined Approximate Bayesian Computation (ABC) in the statistics literature. ABC offers the chance to build upon the progress in probabilistic tomography and overcome downfalls which result from the formalism of likelihood based tomography.\par
This thesis is a 9 month study, undertaken from January 5 - October 14 2018. 
