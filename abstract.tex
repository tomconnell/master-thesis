\chapter{Abstract}

%Geophysical inverse problems open up the interior of the Earth. Their solution describes the inner working of our planet. The methods of inverse problems are a continually evolving field, fed by breakthroughs in mathematics. Recently, probabilistic solutions have been advocated for in geophysics. Probabilistic tomography can tackle the inherit non-uniqueness of the model space and the uncertainty in both data and modelization. This formulation depends on traditional likelihood machinary. By extension, it also suffers from the limiting aspects and assumptions which are implicit in the construction of a likelihood function. Here I explore the geophysical applicability of likelihood-free probabilistic methods, coined Approximate Bayesian Computation (ABC) in the statistics literature. ABC offers the chance to build upon the progress in probabilistic tomography and overcome downfalls which result from the formalism of likelihood based tomography.\par
Geophysical \textit{inverse problems} define Earth's structure based on experiental observations. Their solution is an invaluable line of evidence to scrutinize hypotheses about our planet. Recently, to quantify uncertainty in the experimental observations and the \textit{forward problem}, explore trade-offs between model parameters and constrain the solution with a priori information, probabilistic methods for inverse problems have been pioneered in geophysics. While traditional linear methods are weakened under these conditions probabilistic methods based on a Bayesian formulation are well suited.  There are, however, implicit assumptions in the construction and use of the likelihood function in Bayesian inference. Here I scrutinize these assumptions and advocate for the use of likelihood-free Bayesian inference, coined Approximate Bayesian Computation, as a method which can build upon the success of probabilistic tomography. Here I show that Approximate Bayesian Computation can utilize the information available in a dataset to drive rapid convergence to low misfit models while maintaining formal statistical guarantees. Efficient algorithms of this kind are required for probabilistic methods to tackle the scale and complexity of large inverse problems about Earth’s deep internal structure. By freeing inference from the limiting aspects of a likelihood function, Approximate Bayesian Computation promises to expand parameter inference to problems which were previously intractable. \par
This thesis is a 9 month study, undertaken from January 5 - October 14 2018. 
