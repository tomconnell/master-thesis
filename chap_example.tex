\chapter{ABC}

This chapter will outline a series of illustrative ABC examples. These will serve to demonstrate core concepts touched on in the introduction, demonstrate ABC can sample from complex distributions, and develop the boutique code base which form the foundations of this project. The code for the examples in this section can be found at \url{https://github.com/tomconnell/dram}.\\

\section{Toy problem 1: 1D Gaussian}
As a simple first example consider we have observed $n = 100$ realizations from the Gaussian model $\bm{g_s}(\bm{\theta}) = \frac{1}{\sqrt{2\pi\sigma^2}}\ \text{exp}\Big[\frac{-\mu^2}{2\sigma^2}\Big]$. Our unknown model parameters are $\bm{\theta} = [\mu,\sigma^2]$. Given we have access to simulation from this model it is possible to leverage ABC algorithms to estimate these unknown model parameters. A synthetic dataset for this problem is created with $\mu = 5$ and $\sigma^2 = 4$. The summary statistics of sample mean, $\bar{\mu}$ and sample standard deviation, $\bar{\sigma}$, are used. These provide sufficient statistics with a 1:1 correspondance to the unknown parameters. Figure XX plots the ABC posterior obtained from using algorithm \ref{ABCrejectionsampler}, the traditional form of an ABC rejection sampler, compared to the analytical likelihood. The metric over summary statistics is evaluated marginally and hence takes the form $\rho = |S_1(\bm{y^*}) - S_1(\bm{y})| +| S_2(\bm{y^*}) - S_2(\bm{y})|$. A uniform prior is used to give equal probability to a bounded area, $p(\mu) = \mathcal{U}(-10,10)$ and $p(\sigma^2) = \mathcal{U}(0,100)$. 
Figure YY explores the impact of varying the tolerance for this problem.\\


\section{Toy problem 2: Linear regression}




\section{Toy problem 3: Bivariate Gaussian}




\section{Toy problem 4: Banana distribution}



 